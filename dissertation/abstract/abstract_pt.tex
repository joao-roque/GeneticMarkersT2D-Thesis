% !TEX TS-program = pdflatex
% !TEX encoding = UTF-8 Unicode

\chapter*{Resumo}
\markboth{Resumo}{}

A Diabetes Tipo 2 é uma doença metabólica causada por resistência à insulina nos órgãos, deficiência relativa de insulina, e níveis altos de açucar no sangue. Esta é uma das doenças mais comuns no mundo, e é a quinta maior causa de morte global. Os custos estimados globais de tratamento tanto directo como indirecto, chegam a atingir os US\$1.31 trillion (95\% CI 1.28 - 1.36). Como tal, torna-se cada vez mais important descobrir métodos que possam prever o risco da DT2 desde uma idade jovem, e sem que até nenhuns padrões de risco fisiológicos se verifiquem. Com isto, será possível tanto para médicos como para pacientes estar mais conscientes do risco da doença e poderem empregar medidas preventivas o mais cedo possível.

Existem indícios claros que apontam a Diabetes Tipo 2 como uma patologia influenciada não só por factores ambientais, mas também genéticos. Por isso, este estudo pretende desenvolver novas abordagens a \textit{Genome Wide Association Studies}, mais especificamente no que trata a análises Multi-Locus em doenças complexas, que sejam não só computacionalmente praticáveis mas que estudem também a não-linearidade nestes tipos de dados. Para o fazer, foi desenvolvida uma nova linha inovadora de transformações que permite identificar regiões de interesse no genoma, extrair novas características sem perder o contexto biológico do problema, e utilizá-las em modelos de \textit{Machine Learning} que acontam com a epistasia.

Estes novos métodos são demonstrados numa análise de um dataset de Polimorfismos de Nucleótidos Únicos, onde novos possíveis marcadores genéticos para a Diabetes Tipo 2 são apontados. Para além disso, também é realizada uma classificação do risco de DT2, com \textit{F1-Scores} a atingir os 0.97 com alta confiança. Este projecto pretende sobretudo mostrar como podem ser minados os dados de \textit{datasets} de genótipos de uma maneira que permita o uso de modelos de \textit{Machine Learning} com a sua capacidade total.

\textbf{Palavras-Chave:} Aprendizagem Máquina, Diabetes Tipo 2, Estudos de Associação Genética, Bioinformática, Análise de Dados, Genética.
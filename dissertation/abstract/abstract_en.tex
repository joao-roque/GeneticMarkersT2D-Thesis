% !TEX TS-program = pdflatex
% !TEX encoding = UTF-8 Unicode

\chapter*{Abstract}
\markboth{Abstract}{}

Type 2 Diabetes is a metabolic disorder caused by insulin resistance in organs, relative insulin deficiency and high blood sugar levels. It is one of the most common diseases in the world, and the fifth leading cause of death worldwide, with an estimate global cost of indirect and direct treating reaching US\$1.31 trillion (95\% CI 1.28 - 1.36). As such, it becomes increasingly important to discover methods of predicting T2D risk from a young age and before the onset of any physiological risk patterns, so that both patients and doctors are aware of it, and can monitor the disease and employ preventive measures.

There is clear evidence that supports Type 2 Diabetes risk as being influenced not only by environmental factors, but also genetic ones. In light of this, the following study aims to develop new ways to approach Multi-Locus Genome Wide Association Studies in complex diseases, that are not only computationally feasible, but can also study the non-linearity in a dataset. It aims to do so through the inclusion of an innovative pipeline of transformations that can identify regions of interest in the genome, extract new features without losing biological context of the problem and use them in Machine Learning models that can account for epistasis. 

This process is further demonstrated in an analysis of a Single Nucleotide Polymorphisms dataset, and provides several identifications of possible novel genetic markers for Type 2 Diabetes. Furthermore, classification of T2D's risk is also performed, reaching F1-scores as high as 0.97 with high confidence. This project aims mostly to exhibit how can a genotypes dataset be data mined in a way that can be fully taken advantage of by Machine Learning models.

\textbf{Keywords:} Machine Learning, Type 2 Diabetes, Genome Wide Association Study, Bioinformatics, Data Analysis, Genetics.
% !TEX TS-program = pdflatex
% !TEX encoding = UTF-8 Unicode

% Packages
\usepackage[utf8]{inputenc}
\usepackage[english]{babel}
\usepackage[pdftex]{hyperref}
\usepackage[acronym,toc,style=tree,nonumberlist]{glossaries}
\usepackage[pdftex]{graphicx}
\usepackage{wrapfig}
\usepackage{lipsum}
\usepackage{numprint}
\usepackage[nottoc]{tocbibind}
\usepackage[top=3cm, bottom=3cm, inner=3cm, outer=3cm]{geometry}
\usepackage{acronym}
\usepackage{setspace}
\usepackage{afterpage}
\usepackage{icomma}
\usepackage{calc}
\usepackage{indentfirst}
\usepackage{amsmath}
\usepackage{mathrsfs}
\usepackage{mathbbol}
\usepackage{bm}
\usepackage{tikz}
\usetikzlibrary{datavisualization}
\usetikzlibrary{datavisualization.formats.functions}
\usepackage{fancybox}
\usepackage{booktabs}
\usepackage{dcolumn}
\newcolumntype{d}{D{.}{.}{-1}}
\usepackage{float}
\usepackage{parskip}
\usepackage{fancyhdr}	
\usepackage{eso-pic}	
\usepackage{makeidx}
\usepackage[toc,page]{appendix}

% Options
\hypersetup{
	colorlinks,
	citecolor = black,
	filecolor = black,
	linkcolor = black,
	urlcolor = black,
	pdfstartview = FitH
}

% space between lines
% \linespread{1.6}
\onehalfspacing

% subchapter levels
\setcounter{secnumdepth}{6}
\setcounter{tocdepth}{6}

% defining new commands [simplify work]
\newcommand{\eg}{\emph{e.g. }}
\newcommand{\etal}{\emph{et al. }}
\newcommand{\ie}{\emph{i.e. }}
\DeclareMathOperator{\e}{e}
\DeclareMathOperator{\dd}{d\!}
\newglossary[slg]{symbols}{sym}{sbl}{List of Symbols}

% change some predefined names
\renewcommand{\thefootnote}{\arabic{footnote}}
\renewcommand{\acronymname}{List of Acronyms}
\renewcommand{\bibname}{References}

% define thousand and decimal separators, it is necessary the use of \numprint in order to work well
\npthousandsep{,}\npthousandthpartsep{}\npdecimalsign{.}

% margins
%\let\tmp\oddsidemargin
%\let\oddsidemargin\evensidemargin
%\let\evensidemargin\tmp
%\reversemarginpar

% Create glossaries
\makeglossaries

% Create index
\makeindex

% Chapter title settings
\usepackage{titlesec}		
\titleformat{\chapter}[display]
  {\Huge\bfseries\filcenter}
  {{\fontsize{50pt}{1em}\vspace{-4.2ex}\selectfont \textnormal{\thechapter}}}{1ex}{}[]

% Header and footer settings (Select TWOSIDE or ONESIDE layout below)							
\pagestyle{fancy}  
\renewcommand{\chaptermark}[1]{\markboth{\thechapter.\space#1}{}} 

% Select one-sided (1) or two-sided (2) page numbering
\def\layout{2}	% Choose 1 for one-sided or 2 for two-sided layout
% Conditional expression based on the layout choice
\ifnum\layout=2	% Two-sided
    \fancyhf{}			 						
	\fancyhead[LE,RO]{\nouppercase{ \leftmark}}
	\fancyfoot[LE,RO]{\thepage}
	\fancypagestyle{plain}{			% Redefine the plain page style
		\fancyhf{}
		\renewcommand{\headrulewidth}{0pt} 		
		\fancyfoot[LE,RO]{\thepage}}	
\else			% One-sided  	
  	\fancyhf{}					
	\fancyhead[C]{\nouppercase{ \leftmark}}
	\fancyfoot[C]{\thepage}
\fi

\newcommand{\backgroundpic}[3]{
	\put(#1,#2){
	\parbox[b][\paperheight]{\paperwidth}{
	\centering
	\includegraphics[width=\paperwidth,height=\paperheight,keepaspectratio]{#3}}}}
	
% Caption settings (aligned left with bold name)
\usepackage[labelfont=bf, textfont=normal,
			justification=justified,
			singlelinecheck=false]{caption}